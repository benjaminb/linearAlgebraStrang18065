\documentclass{scrartcl}
\usepackage[T1]{fontenc}
\usepackage{ntheorem}

\usepackage{amsmath}
\DeclareMathOperator{\Res}{Res}
\usepackage{mathpazo}
\usepackage{graphicx}
\newcommand{\R}{\mathbb{R}}

\newcommand*{\vertbar}{\rule[-1ex]{0.5pt}{2.5ex}}
\newcommand*{\horzbar}{\rule[.5ex]{2.5ex}{0.5pt}}

\begin{document}
\title{Linear Algebra and Learning from Data}
\subtitle{Chapter 1}
\author{Selected solutions by Benjamin Basseri}
\date{}
\maketitle

\begin{enumerate}
	\item Give an example where a combination of three nonzero vectors in $\R^4$ is the zero vector. Then write your example in the form $Ax = 0$. What are the shapes of $A, x$ and 0?

We can choose two vectors then make the third vector a linear combination of the first two:
$$\mathbf{a} + \mathbf{b} + (\lambda_1 \mathbf{a} + \lambda_2 \mathbf{b})$$

In this case we may choose:
$$a = \begin{bmatrix}
	1 \\ 0 \\ 0 \\ 0
\end{bmatrix}, 
b = \begin{bmatrix}
	0 \\ 1 \\ 0 \\ 0
\end{bmatrix},
c = \begin{bmatrix}
	1 \\ 1\\ 0 \\ 0
\end{bmatrix}$$

Then $a + b - c = 0$. This linear combination of $a, b, c$ can be expressed in matrix form as:
$$[a \ b \ c] \begin{bmatrix}
	1 \\ 1 \\ -1
\end{bmatrix} = 
\begin{bmatrix}
	1 & 0 & 1 \\
	0 & 1 & 1 \\
	0 & 0 & 0 \\ 
	0 & 0 & 0 \\
\end{bmatrix}
\begin{bmatrix}
	1 \\ 1 \\ -1
\end{bmatrix} = \mathbf{0}$$

where $A$ is $4 \times 3$, $x$ is $3 \times 1$, and 0 is $3\times 1$.

\item Suppose a combination of the columns of $A$ equals a different combination of those columns. Write that as $Ax = Ay$. Find two combinations of the columns of $A$ that equal the zero vector (in matrix language, find two solutions to $Az = 0$).

$$Ax = Ay \implies Ax - Ay = 0 \implies A(x-y) = 0 \implies x-y \in \ker A$$

This also implies $y-x$ is in the null space of $A$ since we could subtract $Ax$ from both sides in the first step just as well.

\item The vectors $\mathbf{a}_1, \mathbf{a}_2, \ldots, \mathbf{a}_n$ are in $m$-dimensional space $\R^m$, and a combination $c_1\mathbf{a}_1 + \ldots + c_n \mathbf{a}_n$ is the zero vector. That statement is at the vector level.
\begin{enumerate}
	\item Write that statement at the matrix level. Use the matrix $A$ with the $\mathbf{a}$'s in its columns and use the column vector $\mathbf{c} = (c_1, \ldots, c_n)$.

One of Strang's pictures of matrix multiplication describes $A\mathbf{c}$ as a linear combination of the columns of $A$, with the combination given by the components of $\mathbf{c}$. Thus
\[
A\mathbf{c} = 
\left[
  \begin{array}{cccc}
    \vertbar & \vertbar &        & \vertbar \\
    \mathbf{a}_{1}    & \mathbf{a}_{2}    & \ldots & \mathbf{a}_{n}    \\
    \vertbar & \vertbar &        & \vertbar 
  \end{array}
\right]\begin{bmatrix}
	c_1 \\ \vdots \\ c_n
\end{bmatrix} = 0
\]
$$\iff c_1 \mathbf{a}_1 + \ldots c_n \mathbf{a}_n = 0$$


	\item Write that statement at the scalar level. Use subscripts and sigma notation to add up numbers. The column vector $\mathbf{a}_j$ has components $a_{1j}, a_{2j}, \ldots, a_{mj}$.

From the matrix picture we can write out in summation for the $j$th component of $A\mathbf{c}$:
$$(A\mathbf{c})_j = \sum_{i=0}^m c_i a_{ji} = 0$$
$$\text{altogether: } A\mathbf{c} = \sum_{j=1}^n \sum_{i=1}^m c_i a_{ij}\mathbf{e}_j = \mathbf{0}$$
\end{enumerate}

\item Suppose $A$ is the 3 by 3 matrix of all ones. Find two independent vectors $x$ and $y$ that solve $Ax = 0$ and $Ay = 0$. Write that first equation $Ax = 0$ (with numbers) as a combination of the columns of $A$. Why don't I ask fo ra third independent vector with $Az = 0$?

We can choose $x = (1, -1, 0), y = (0, 1, -1)$. These are independent since clearly there is no scalar $\lambda$ such that $\lambda x = y$: the scalar $\lambda $ would have to map $x_1 \mapsto 0$ which implies $\lambda = 0$, but then $x_2 \mapsto 1$ which implies $\lambda \neq 0$. 

Let $\mathbf{1}$ be the vector of 1's, which in this case is each column of $A$. Then
$$Ax = \sum_{i=1}^3 x_i \mathbf{1} = x_1 \mathbf{1} + x_2 \mathbf{1} + x_3\mathbf{1} = 1\cdot \mathbf{1} - 1\cdot\mathbf{1} + 0 = 0$$

There is no 3rd independent vector in the null space of $A$. This follows from rank-nullity: $A$ has a rank of 1, its column space is the subspace spanned by $\mathbf{1}$ (a line). Since we're in $\R^3$, that means $N(A)$ has dimension 2.

\item The linear combinations of $v =(1, 1, 0)$ and $w = (0, 1, 1)$ fill a plane in $\R^3$.
\begin{enumerate}
	\item Find a vector $z$ that is perpendicular to $Wv$ and $w$. 

Such a vector $z$ must have a 0 dot product with $v$ and $w$. By inspection we see that if $z = (1, -1, 1)$ then the dot product with either $v$ or $w$ will get a 1 from the first or third component and a -1 from the second:

$$z \cdot v = \begin{bmatrix}
	1 \\ -1 \\ 1
\end{bmatrix}\cdot
\begin{bmatrix}
	1 \\ 1 \\0
\end{bmatrix} = 0$$
$$z \cdot w = \begin{bmatrix}
	1 \\ -1 \\ 1
\end{bmatrix}\cdot
\begin{bmatrix}
	0 \\ 1 \\ 1
\end{bmatrix} = 0$$

\item Find a vector $u$ that is not on the plane. Check that $u^\top z \neq 0$


	Of course $z$ is not in the plane and since $z \neq 0$, we have $z^\top z \neq 0$.
\end{enumerate}

\item If three corners of a parallelogram are (1,1), (4,2), and (1,3), what are all three of the possible fourth corners? Draw two of them.

By the parallelogram rule of vector addition, we can take the vector represented by any point and add the vectors of the other two points' \emph{differences} to get the fourth point of the parallelogram. Starting at $(1,1)$, we have differences to $(1, 3)$ and $(4,2)$ as follows:
$$(1,3) - (1,1) = (0, 2), \quad (4,2) - (1,1) = (3,1)$$

so a point that makes a parallelogram would be (1,1) + (0, 2) + (3,1) = (4, 4).
\begin{center}
	\includegraphics[width=0.5\textwidth]{imgs/1.1.6.png}
\end{center}
\end{enumerate}

\end{document}